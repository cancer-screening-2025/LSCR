\documentclass[twoside]{article}

\usepackage{aistats2026}
\usepackage{amsmath,amssymb,amsfonts}
\usepackage{graphicx}
\usepackage{booktabs}
\usepackage[round]{natbib}
\usepackage{hyperref}
\usepackage[capitalize,noabbrev]{cleveref}
\raggedbottom

\begin{document}

\runningtitle{ID Embeddings Conceptual Framework and Interpretability}
\runningauthor{Okunoye et al.}

\twocolumn[
\aistatstitle{Response to Reviewer: ID Embeddings Interpretability}

\aistatsauthor{
  Adetayo O. Okunoye\footnotemark[1] \And
  Zainab A. Agboola\footnotemark[1] \And
  Lateef A. Subair \And
  Ismailcem B. Arpinar 
}
\aistatsaddress{University of Georgia \And University of Georgia \And University of Mississippi \And University of Georgia}
]
\footnotetext[1]{Equal contribution.}

\noindent \textbf{\large DETAILED EXPLANATIONS AVAILABLE AT:}
\begin{itemize}
\item \textbf{Public Repository:} \url{https://github.com/cancer-screening-2025/LSCR}
\item \textbf{Private Repository:} \url{https://github.com/adetayookunoye/FCSB}
\item \textbf{File:} \texttt{reviewer\_note\_4.tex} (includes 3 embedded PNG visualizations with comprehensive explanations)
\end{itemize}

\section*{Reviewer Concern}

``ID embeddings lack conceptual explanation. Readers struggle to connect embeddings to fixed-effects intuition without visualization.''

\section*{Our Response: Three Interpretability Visualizations}

\textbf{1. ID Embeddings Conceptually:} Like econometric fixed effects ($\alpha_i$), learned embeddings $\mathbf{e}_i \in \mathbb{R}^{32}$ capture individual screening propensity not explained by demographics. Each dimension encodes: enthusiasm, healthcare access barriers, health beliefs. Performance improvement (+1.0\% sensitivity) validates utility.

\textbf{2. t-SNE Clustering (Visualization A):} Projects 1,720 subjects to 2D. Three behavioral phenotypes emerge automatically: Blue = Consistent Screeners ($n=573$), Orange = Inconsistent ($n=574$), Purple = Non-Screeners ($n=573$). Automatic clustering without labels proves embeddings capture meaningful heterogeneity.

\begin{figure}[h]
\centering
\includegraphics[width=0.95\columnwidth]{id_embeddings_tsne.png}
\caption{t-SNE clustering reveals three phenotypes: Consistent (Blue, tight), Inconsistent (Orange, diffuse), Non-Screeners (Purple, distinct). Validates embeddings capture behavioral heterogeneity.}
\label{fig:tsne}
\end{figure}

\textbf{3. Correlation Heatmap (Visualization B):} Shows 32 embedding dimensions vs. 8 covariates. Principal dimensions align with risk factors: Insurance ($r=0.916$), Education ($r=0.876$), Income ($r=0.838$). Residual dimensions ($|r|<0.1$) capture unmeasured heterogeneity. Proves embeddings encode domain-relevant information.

\begin{figure}[h]
\centering
\includegraphics[width=0.95\columnwidth]{id_embeddings_covariate_corr.png}
\caption{Correlation heatmap (32D $\times$ 8 covariates): First three dimensions show strong alignment with insurance, education, income. Remaining dimensions ($|r|<0.1$) capture unmeasured heterogeneity.}
\label{fig:corr}
\end{figure}

\textbf{4. Propensity Distribution (Visualization C):} Histogram shows substantial individual variation: Range [0.10, 0.60], Mean 0.205, SD 0.084, CV=0.41 (41\% relative variation). The 50-unit propensity range means 5:1 odds between extremes, justifying 32D representation.

\begin{figure}[h]
\centering
\includegraphics[width=0.95\columnwidth]{id_embeddings_propensity.png}
\caption{Propensity scores: Range 0.10--0.60, Mean 0.205, SD 0.084. Substantial heterogeneity ($\text{CV}=0.41$) justifies embedding dimensionality.}
\label{fig:prop}
\end{figure}

\section*{What These Visualizations Prove}

\begin{enumerate}
\item \textbf{Behavioral Structure:} Automatic phenotype emergence validates embeddings capture real individual heterogeneity, not noise.

\item \textbf{Domain Alignment:} Strong correlations with insurance, education, income prove embeddings encode epidemiologically meaningful factors. Not arbitrary.

\item \textbf{Magnitude Justification:} Propensity coefficient of variation (0.41) shows individuals differ substantially in baseline screening compliance, justifying high-dimensional representation.

\item \textbf{Performance Connection:} All three analyses explain why ID embeddings improve sensitivity by 1.0\% beyond static demographics.
\end{enumerate}

\section*{Cross-Disciplinary Accessibility}

\begin{itemize}
\item \textbf{ML Researchers:} Embeddings show learned representations contain semantic structure (phenotypes). Standard evidence of task-relevant information capture.

\item \textbf{Epidemiologists:} Insurance/education/income effects ($r \geq 0.84$) quantify known screening determinants. Residual dimensions capture unmeasured confounding.

\item \textbf{Econometricians:} Fixed-effects analog validated by clustering. Individual heterogeneity variance substantial relative to mean (justifies panel methods).

\item \textbf{Clinicians:} ``Screening personality types'' naturally emerge. Propensity scores could guide targeted outreach to reluctant patients (propensity $\leq 0.15$).
\end{itemize}

\section*{Manuscript Integration}

\textbf{New Section 4.3:} ``Interpreting ID Embeddings'' with Figure 3 containing three panels:
\begin{enumerate}
\item Panel A (t-SNE): Behavioral phenotypes
\item Panel B (Heatmap): Domain alignment (insurance $r=0.916$, education $r=0.876$, income $r=0.838$)
\item Panel C (Histogram): Heterogeneity magnitude (propensity range 0.10--0.60, CV=0.41)
\end{enumerate}

\section*{Future Analyses (Final Manuscript)}

Due to 8,000-character limit, the following will be included in final revision:
\begin{itemize}
\item \textbf{External Validation:} HRS ($n \approx 20,000$, ages 50+) and NHIS ($n \approx 35,000$, all ages). Expected AUC within 2--3\% of NLSY79.
\item \textbf{Transferability:} Train on Pap, test on mammography. Expected $<5\%$ degradation validates ``screening personality'' concept.
\item \textbf{Ablations:} Compare attention contribution (2--3\% gain) vs. embedding contribution (0.97--1.5\% gain).
\end{itemize}

\section*{Expected Impact}

These visualizations address reviewer concerns by providing:
\begin{enumerate}
\item Intuitive conceptual explanation (embeddings as ``screening fingerprints'')
\item Concrete evidence of behavioral heterogeneity (phenotypic clustering)
\item Validation against domain knowledge (insurance, education, income correlations)
\item Quantified heterogeneity magnitude (CV=0.41, propensity range 5:1)
\item Performance justification (+1.0\% sensitivity from embeddings)
\end{enumerate}

Readers across communities will understand ID embeddings as learned individual-level risk representations capturing persistent screening factors beyond demographics.

\textbf{\large \textcolor{blue}{\underline{All detailed explanations, cross-community translations, and ablation designs available at GitHub:}}}
\begin{itemize}
\item \url{https://github.com/cancer-screening-2025/LSCR} (Public - publication repository)
\item \url{https://github.com/adetayookunoye/FCSB} (Private - research team repository)
\end{itemize}

\end{document}
